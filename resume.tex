\documentclass{article}
\usepackage[utf8]{inputenc}
% \usepackage[a4paper, total={6in, 8in}]{geometry}
\usepackage[left=0.5in,top=0.6in,right=0.5in,bottom=0.6in]{geometry}
\usepackage{enumitem}
\usepackage{hyperref}
\hypersetup{
    colorlinks=true,
    linkcolor=blue,
    filecolor=magenta,      
    urlcolor=cyan,
    pdftitle={Overleaf Example},
    pdfpagemode=FullScreen,
    }

\begin{document}

\begin{center}
{\huge \textbf{Collin Zhang}}\par
rz1477@nyu.edu\par(917) 621-1855
\end{center}

\noindent
{\textbf{EDUCATION}}\newline
\rule{\textwidth}{1pt}
{\textbf {New York University}}\newline
\emph {B.S. Computer Science; B.S., Business}.\newline
{Sep 2019 - Jun 2023, New York}\newline
\newline
\noindent
{\textbf{PROJECT EXPERIENCE}}\newline
\rule{\textwidth}{1pt}\newline{\textbf{CoCreate}}\newline
\emph{Team Leader (Oct 2019 - June 2021)}
\begin{itemize}[leftmargin=*,topsep=0pt]
\item An iOS app integrating real-time collaboration with hand-writing drawing board based on Apple Pencil
\item Use of Flask and Socket.IO with python to build a backend that supports real-time collaboration of drawing
\item On the frontend, design an algorithm with Swift that converts the points produced by users(location and force) to segments of bezier curve that produce a smooth hand-writing curve
\item To achieve lasso on a drawing of thousands of curves, design a map that divides the drawing to 10*10 sqaures, indexing segments of curves to each sqaure, create an algorithm that determines squares covered by an arbitrary self-intersecting polygon, and an algorithm using ray casting algorithm to determine whether a segment is in it, achieve 100 times speed up
\item Abstracing a drawing module FastDraw for modularity, providing a DrawBoardView for drawing, delegation is used to signal operation like drawing, erasing and lasso, so operation can be sent to server for collaboration
\item 9.59k downloads on App Store, 9400+ boards created by users
\item \url{https://apps.apple.com/tt/app/cocreate-draw-together/id1548911886?ign-mpt=uo%3D2}

\end{itemize} \ \\ {\textbf{Covid-19 abnormalities detection on chest radiographs}}\newline
\emph{July 2021 - August 2021}
\begin{itemize}[leftmargin=*,topsep=0pt]
\item Build a model to detect abnormal area and determine if the patient has Covid-19
\item Preprocess data with pandas, visualize the radiographs with matplotlib and pydicom, performing EDA
\item Train a Cascade R-CNN for bounding box regression with mmdetection
\item Train a EfficientNet for classification with keras

\end{itemize} \ \\ 
\noindent
{\textbf{INTERNSHIP EXPERIENCE}}\newline
\rule{\textwidth}{1pt}\newline{\textbf{Supersymmetry}}\newline
\emph{iOS Developer (Jan 2021 - May 2021)}
\begin{itemize}[leftmargin=*,topsep=0pt]
\item Participate in developing a social media app: Project Z
\item Technology Stack: Swift, RxSwift, Starscream, Moya
\item Write reusable UI components with SnapKit, Use of Moya to abstract the api call for understandable code
\item Discuss with colleagues from backends for implementation of business logic which is efficient, easier to implement and supports backward compatibility

\end{itemize} \ \\ 
\noindent
{\textbf{SKILLS AND OTHERS}}\newline
\rule{\textwidth}{1pt}\begin{itemize}[leftmargin=*,topsep=0pt]
\item Courses: Data Structures, Computer System Organization, Intro to Algorithms, Discrete Math
\item Swift, Python, Java, C++, Javascript, SQL, HTML/CSS, Tensorflow, Keras
\item Contribution to Tensorflow, Keras, and mmdetection on Github

\end{itemize} \ \\ \end{document}