\documentclass{article}
\usepackage[utf8]{inputenc}
% \usepackage[a4paper, total={6in, 8in}]{geometry}
\usepackage[left=0.5in,top=0.6in,right=0.5in,bottom=0.6in]{geometry}
\usepackage{enumitem}
\usepackage{hyperref}
\hypersetup{
    colorlinks=true,
    linkcolor=blue,
    filecolor=magenta,      
    urlcolor=cyan,
    pdftitle={Overleaf Example},
    pdfpagemode=FullScreen,
    }

\begin{document}

\begin{center}
{\huge \textbf{Collin Zhang}}\par
rz1477@nyu.edu\par+86 13395122259
\end{center}

\noindent
{\textbf{EDUCATION}}\newline
\rule{\textwidth}{1pt}
{\textbf {New York University}}\newline
\emph {B.S. Computer Science; B.S., Business}.\newline
{Sep 2019 - Jun 2023, New York}\newline
\newline{\textbf {Nanjing Foreign Language School}}\newline
{Sep 2016 - Jun 2019, Nanjing}\newline
\newline
\noindent
{\textbf{PROJECT EXPERIENCE}}\newline
\rule{\textwidth}{1pt}\newline{\textbf{CoCreate}}\newline
\emph{Team Leader (Oct 2019 - June 2021)}
\begin{itemize}[leftmargin=*,topsep=0pt]
\item An iOS app integrating real-time collaboration with hand-writing drawing board based on Apple Pencil
\item Design of data structure and search algorithm to realize high performance write, read, search, and update (includes pen, eraser and lasso) function, and algorithm to improve performance of Apple Pencil writing
\item Technology stack: Swift, SQLite, MVC Architecture, Socket.IO, Flask, Gunicorn, Nginx
\item Using Github and Lark to set goals and collaborate with team members
\item \url{https://apps.apple.com/tt/app/cocreate-draw-together/id1548911886?ign-mpt=uo%3D2}

\end{itemize} \ \\ {\textbf{Covid-19 abnormalities detection on chest radiographs}}\newline
\emph{July 2021 - August 2021}
\begin{itemize}[leftmargin=*,topsep=0pt]
\item Build a model to detect abnormal area and determine if the patient has Covid-19
\item Preprocess data with pandas, visualize the radiographs with matplotlib and pydicom, performing EDA
\item Train a Cascade R-CNN for bounding box regression with mmdetection
\item Train a EfficientNet for classification with keras

\end{itemize} \ \\ {\textbf{ItemMemo}}\newline
\emph{Developer (Sep 2020 - Oct 2020)}
\begin{itemize}[leftmargin=*,topsep=0pt]
\item An iOS app that helps users stop losing things
\item Technology stack: WidgetKit, WatchKit and SwiftUI
\item Application of Growth Hacking: cooperate with KOLs on Social Media to achieve fast growth of downloads; campaigns on Facebook Ads to attract users from United States
\item Got a downloads of 11k around the world
\item \url{https://apps.apple.com/cn/app/id1531614603}

\end{itemize} \ \\ 
\noindent
{\textbf{INTERNSHIP EXPERIENCE}}\newline
\rule{\textwidth}{1pt}\newline{\textbf{Supersymmetry}}\newline
\emph{iOS Developer (Jan 2021 - May 2021)}
\begin{itemize}[leftmargin=*,topsep=0pt]
\item Participate in developing a social media app: Project Z
\item Technology Stack: Swift, RxSwift, Starscream, Moya
\item Write reusable UI components with SnapKit, Use of Moya to abstract the api call for understandable code
\item Discuss with colleagues from backends for implementation of business logic which is efficient, easier to implement and supports backward compatibility

\end{itemize} \ \\ 
\noindent
{\textbf{SKILLS AND OTHERS}}\newline
\rule{\textwidth}{1pt}\begin{itemize}[leftmargin=*,topsep=0pt]
\item Bilingual in Chinese and English
\item Programming languages: Swift, Python, Java, C++, Javascript, SQL, HTML/CSS
\item Contribution to Tensorflow, Keras, and mmdetection on Github

\end{itemize} \ \\ \end{document}